\begin{table}[ht!]
\centering
\caption{Robustness of Staggered DiD Estimates}
\label{tab:robustness}
\begin{threeparttable}
\begin{tabular}{l*{5}{c}}
\toprule
 & \multicolumn{4}{c}{Urban--rural income ratio} & Log ratio \\
\cmidrule(lr){2-5} \cmidrule(lr){6-6}
 & Excl.\ 2009 & 2011+ & Excl.\ 2011 & Extended & Log \\
 & cohort & cohorts & cohort & controls & spec. \\
 & (1) & (2) & (3) & (4) & (5) \\
\midrule
Treated ($D_{it}$) & $-0.094^{**}$ & $-0.129^{***}$ & $-0.090^{**}$ & $-0.066^{**}$ & $-0.016^{*}$ \\
 & $(0.041)$ & $(0.043)$ & $(0.041)$ & $(0.030)$ & $(0.009)$ \\
\midrule
Controls & No & No & No & Yes & Yes \\
County FE & Yes & Yes & Yes & Yes & Yes \\
Year FE & Yes & Yes & Yes & Yes & Yes \\
Observations & 276 & 204 & 348 & 444 & 444 \\
Clusters & 23 & 17 & 29 & 37 & 37 \\
$R^2$ (within) & 0.098 & 0.221 & 0.079 & 0.169 & 0.051 \\
\bottomrule
\end{tabular}
\begin{tablenotes}[flushleft]
\small
\item \textit{Notes:} Columns~(1)--(3) use binary treatment without controls on restricted samples. Column~(4) augments baseline controls with log population and primary-sector GDP share. Column~(5) uses the natural log of the income ratio as the dependent variable, with baseline controls. Standard errors clustered at the county level in parentheses. $^{*}p<0.1$; $^{**}p<0.05$; $^{***}p<0.01$.
\end{tablenotes}
\end{threeparttable}
\end{table}
